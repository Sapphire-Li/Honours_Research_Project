
\section{Background}


\begin{frame}
    \frametitle{Explanations of the puzzle in literature}

        \begin{exampleblock}{\large{Uncertainty in Weight Estimation}}
        The simple averaging does not require any estimation (\cite{SW98}, \cite{SW04}, and \cite{SW09}).
        \end{exampleblock}

        \vspace{4mm}

        \begin{exampleblock}{\large{Trade-off between Bias and Variance}}
        The equally weighted combination is unbiased and its variance has only one component (\cite{E11} and \cite{CMVW16}).
        \end{exampleblock}

\end{frame}




\begin{frame}
\frametitle{A Recent (General) Explanation}

    \begin{exampleblock}{\large{Estimation Uncertainty on Forecast Performance}}
    Asymptotically, the bias and sampling variability mainly come from the estimation of the models used to produce the constituent model forecasts (\cite{ZMFP22} and \cite{FZMP23}).
    \end{exampleblock}

    \vspace{4mm}
    
    These explanations all implicitly assume that \textbf{the puzzle will be in evidence} when combining forecasts, regardless of the choice of constituent models or the weighing scheme. 


\end{frame}


\begin{frame}{Research Gap}
    
    Forecast combination has attracted wide attention and contributions in the literature, both theoretical and applied (\cite{C89} and \cite{T06}).

    \vspace{4mm}
    
    Researchers have examined a variety of combination methods for both point and density forecasts over the past 50 years, see \cite{WHLK22} for a modern literature review.

    \vspace{4mm}
    
    No attention appears to have been given to \textbf{the cross-sectional setting}. 
    
\end{frame}


\begin{frame}
    \frametitle{Research objectives}

    \begin{enumerate}[<+->]
        \item To \textbf{substantiate the presence} of the combination puzzle in the time series setting with empirical data. \newline
        \item To systematically investigate the determinants behind, and evidence for, the \textbf{forecast combination puzzle in the cross-sectional setting} using simulated data. \newline
        \item To \textbf{validate our preliminary conjecture} with empirical evidence.
    \end{enumerate}

\end{frame}

