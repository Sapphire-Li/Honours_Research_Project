
\section{Methodology}

\begin{frame}{Forecast Combination}

    We focus on the combination of forecasts from non-nested models for a given dataset, which is commonly performed in two stages:

    \vspace{5mm}

    \begin{enumerate}[<+->]
        \item \textbf{producing} separate point or probabilistic \textbf{forecasts} for the next time point using observed data and constituent models, and \newline
        \item \textbf{combining forecasts} based on one of the accuracy criteria. 
    \end{enumerate}

\end{frame}



\begin{frame}{First Stage - Parameters Estimation}

    The unknown parameters of each model, $\theta$, are estimated by maximizing the log likelihood function over the in-sample period ($R$):


    \begin{equation}
    \label{eqn:theta}
    \hat\theta = \underset{\theta}{\text{argmax}} \sum^R_{t=1} log \ f(y_t).
    \end{equation}

    \begin{itemize}
        \item Point forecasts
        \item Density forecasts
    \end{itemize}


\end{frame}



\begin{frame}{Point: Linear Combination}
    
    Two constituent points, $y_{1t}$ and $y_{2t}$, are aggregated linearly:
    
    \vspace{3mm}
    
    \begin{equation}
    \label{eqn:PC1}
    y_t = \omega \ y_{1t} + (1-\omega) \ y_{2t}
    \end{equation}
    
    \vspace{3mm}
    
    where $\omega$ is the non-negative weight allocated to the point expressed based on the first model.

\end{frame}



\begin{frame}{Density: Linear Pools}
		
        A linear combination of two densities, $f(y_t)$, is constructed with two constituent densities $f_1(y_t)$ and $f_2(y_t)$:
        
        \vspace{3mm}
        
        \begin{equation}
        f(y_t) = \omega \ f_1(y_t) + (1-\omega) \ f_2(y_t)
        \end{equation}  
        
        \vspace{3mm}
        
        where $\omega$ is the weight allocated to the first density.

        \vspace{2mm}

        \small{The sum of two weights is implied to be 1, which is necessary and sufficient for the combination to be a density function [\cite{GA11}].}
        
\end{frame}



\begin{frame}{Second Stage - Weight Estimation}

    The optimal weight assigned to the first density is estimated by maximizing the log score function over the in-sample period:

    \begin{equation}
    \hat{\omega}_{\text{opt}} = \underset{\omega}{\text{argmax}} \sum^R_{t=1}log \Big[\omega \ f_1(y_t |\mathcal{F}_{\small{t-1}}, \hat\theta_1) + (1-\omega) \ f_2(y_t|\mathcal{F}_{\small{t-1}}, \hat\theta_2)\Big].
    \end{equation}

    \begin{itemize}
    \small{
    \item $R$ = the in-sample period
    \item $\mathcal{F}_{\small{t-1}}$ = all information available at time $t-1$
    }
    \end{itemize}

\end{frame}



\begin{frame}{Mean Squared Forecast Error}
    The Mean squared forecast error (MSFE) of an individual model is the average squared difference between the actual value, $y_t$, and the predicted value, $\hat y_t$, at each time point over the in-sample period $R$:

    \begin{equation}
    \label{eqn:MSFE1}
    MSFE = \frac{1}{R} \sum^R_{t=1} (y_t - \hat y_t)^2.
    \end{equation}  

\end{frame}



\begin{frame}{Log Predictive Score Function}

\small{Following \cite{GA11}, we measure the accuracy of the density forecasts with the log predictive score function over the forecast horizon $h=1,2,...,P$ (i.e., the out-of-sample period), which is defined as:}

\vspace{-2mm}

\begin{equation}
    LS = \sum^P_{h=1} log \Big[ \hat{\omega}_{\text{opt}} f_1(y_{\small{R+h}}| \mathcal{F}_{\small{R+h-1}}, \hat\theta_1) + (1- \hat{\omega}_{\text{opt}}) \ f_2(y_{\small{R+h}}| \mathcal{F}_{\small{R+h-1}}, \hat\theta_2)\Big].
\end{equation}


\vspace{2mm}

\begin{itemize}
\small{
    \item $P$ = the out-of-sample period
    \item $h$ = the forecast horizon ($h>0$)
    \item $\mathcal{F}_{\small{R+h-1}}$ = all information available at time ($R+h-1$)
    \item $\hat\theta$ = parameter estimates
    }
\end{itemize}

\end{frame}



