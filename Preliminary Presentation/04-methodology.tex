
\section{Methodology}


\begin{frame}
\label{Figure}
	\frametitle{Forecast combination method}
		
        A linear combination of two predictive densities, $f(y_t)$, is constructed with two constituent predictive densities $f_1(y_t)$ and $f_2(y_t)$:
        
        \vspace{3mm}
        
        \begin{equation}
        f(y_t) = w \ f_1(y_t) + (1-w) \ f_2(y_t)
        \end{equation}  
        
        \vspace{5mm}
        
        where $w$ \footnote{\scriptsize{Through this construction, the sum of two weights is implied to be 1, which is necessary and sufficient for the combination to be a density function [\cite{GA11}].}} is the weight allocated to the first model. 
        
\end{frame}


\begin{frame}
\frametitle{Parameters Estimation}

The unknown parameters, $\theta_M$, of each model ($M$) are estimated by maximizing the log likelihood function of the conditional probability density over the in-sample period ($R$):

\begin{equation}
    \hat\theta_M = \text{argmax} \sum^R_{t=1} log f(y_t|M).
\end{equation}
 
\end{frame}

\begin{frame}{Log predictive score function}

Following \cite{GA11}, we measure the accuracy of density forecasts with the log predictive score function for each model, which is defined as:

\begin{equation}
LS = \sum^P_{h=1} log\ f(y_{\small{R+h}}| \mathcal{F}_{\small{R+h-1}}, \hat\theta_M, M)
\end{equation}

\vspace{5mm}
\scriptsize{where $P$ denotes the out-of-sample period and $h$ is the forecast horizon with $h>0$. $\mathcal{F}_{\small{R+h-1}}$ denotes all information available at time $R+h-1$, and we assume that the conditional mean and variance of the models are, up to unknown parameters, known at time $R+h-1$. }

\end{frame}

\begin{frame}{Weight Estimation}
    The weight ($w$) assigned to the first model will be estimated by maximizing the log predictive score function over the out-of-sample period:

\begin{multline}
\hat{w} = \text{argmax} \sum^P_{h=1}log\Big[w \ f_1(y_{\small{R+h}}|\mathcal{F}_{\small{R+h-1}}, \hat\theta_{M_1}, M_1) \\
 + (1-w) \ f_2(y_{\small{R+h}}|\mathcal{F}_{\small{R+h-1}}, \hat\theta_{M_2}, M_2)\Big]
\end{multline}

\end{frame}


